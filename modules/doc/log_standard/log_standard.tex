

\documentclass[paper=letter, fontsize=10pt]{scrartcl} % Letter paper and 11pt font size
\usepackage{graphicx}
\usepackage[space]{grffile}

\listfiles

\usepackage[T1]{fontenc} % Use 8-bit encoding that has 256 glyphs
\usepackage{fourier} % Use the Adobe Utopia font for the document - comment this line to return to the LaTeX default
\usepackage[english]{babel} % English language/hyphenation
\usepackage{amsmath,amsfonts,amsthm} % Math packages
\usepackage{gensymb}
\usepackage{textcomp}
\usepackage{comment}
\usepackage{enumerate}
\usepackage{setspace}
\usepackage{color}
\usepackage{multirow}

\usepackage{listings}

\usepackage{bytefield}
\definecolor{lightgray}{gray}{0.8}
\newcommand{\bfbitwidth}{1.3em}

% From http://tex.stackexchange.com/a/24133
\makeatletter
\newcommand*{\textoverline}[1]{$\overline{\hbox{#1}}\m@th$}
\makeatother

\usepackage[usenames,dvipsnames]{xcolor}

\usepackage{sectsty} % Allows customizing section commands
\allsectionsfont{\centering \normalfont \scshape} % Make all sections centered, the default font and small caps
\subsectionfont{\normalfont} %Left-justify subsections
\subsubsectionfont{\normalfont} %Left-justify subsections
\usepackage{titlesec}
\titleformat{\subsection}[runin] {\normalfont\large\bfseries}{\thesubsection}{1em}{}
\titleformat{\subsubsection}[runin] {\normalfont\bfseries}{\thesubsubsection}{1em}{}


\setlength{\parskip}{8pt plus 5pt minus 3pt}

%\pagenumbering{gobble}

\numberwithin{equation}{section} % Number equations within sections (i.e. 1.1, 1.2, 2.1, 2.2 instead of 1, 2, 3, 4)
\numberwithin{figure}{section} % Number figures within sections (i.e. 1.1, 1.2, 2.1, 2.2 instead of 1, 2, 3, 4)
\numberwithin{table}{section} % Number tables within sections (i.e. 1.1, 1.2, 2.1, 2.2 instead of 1, 2, 3, 4)

\setlength{\parindent}{0pt} % Removes all indentation from paragraphs - comment this line for an assignment with lots of text

\usepackage[margin=0.85in]{geometry}

\usepackage{fancyhdr} % Custom headers and footers
\pagestyle{fancyplain} % Makes all pages in the document conform to the custom headers and footers
\fancyhead{} % No page header
%\fancyhead[L]{ \scshape{\textit{\Large{\textsc{\color{Gray}{Confidential}}}}} \hfill \scriptsize{\color{gray}{\rev}}}  
\fancyfoot[L]{ \thepage |  Page} % Empty left footer
\fancyfoot[C]{} % Empty center footer
\fancyfoot[R]{\textsc{Prepared on {\today}}} % Page numbering for right footer
\renewcommand{\headrulewidth}{0pt} % Remove header underlines
\renewcommand{\footrulewidth}{1pt} % Comment out to Remove footer underlines
	\renewcommand{\footrule}{{\color{gray}%
	\vskip-\footruleskip\vskip-\footrulewidth
	\hrule width\headwidth height\footrulewidth\vskip\footruleskip}}
\setlength{\headheight}{13.4pt} % Customize the height of the header

\usepackage{tabularx}
\usepackage{booktabs} %allows use of excel tables in latex
\usepackage{pdfpages}
\usepackage{csquotes}

\usepackage{hyperref}
\begin{document}

\renewcommand{\arraystretch}{1.5}

\newcommand{\hreffoot}[2]{\href{#1}{#2}\footnote{#1}}
\newcommand{\horrule}[1]{\rule{\linewidth}{#1}} % Create horizontal rule command with 1 argument of height
\normalfont \normalsize 
{\hfill
\includegraphics[scale=0.1]{images/bexigon_light.png} \\[1pt] % Your university, school and/or department name(s)
}
\horrule{0.5pt} \\[.2cm] % Thin top horizontal scrule
{\centering 
	\sc \LARGE Logging Format Standard \\
    \sc Boulder Engineering Studio Internal\\
}
\horrule{0.5pt} \\[0.1cm] % Thick bottom horizontal rule
\\ [5pt]
\\  [5pt]

\begin{table}[ht!]
    \begin{center}
        \begin{tabularx}{\textwidth}{|l|l|X|}
            \hline
            \emph{Date} & \emph{Author} & \emph{Notes}                                                               \\ \hline
            2016-01-15  & Austin Glaser & Initial Draft                                                              \\ \hline
            2016-03-12  & Austin Glaser & Add 'Runtime Error' log type                                               \\ \hline
            2016-03-14  & Austin Glaser & Modify 'Entry Length' field definition, modify padding length field widths \\ \hline
            2016-03-29  & Austin Glaser & Add new logo, update title block                                           \\ \hline
        \end{tabularx}
        \caption{Revision History.}
        \label{tab:revision}
    \end{center}
\end{table}

\tableofcontents

\clearpage

\section{Introduction} \label{sec:introduction}

\subsection{Scope} \label{sec:scope}

This document specifies a standard data format for logging events to an
arbitrary section of nonvolatile memory. It does not specify the type of memory,
or the interface protocol. It also does not specify the software API used -- for
that, consult either a separate document or the software's internal
documentation.

\subsection{Purpose} \label{sec:purpose}

This data format is intended to be used as a standard across all BES projects
where some form of logging is desirable, but which are not high-level enough to
include an on board filesystem with high-capacity, persistent storage. It is
intended to be simple to implement and consume little memory relative to a
'standard' logging protocol which saves raw text.

It's also intended to be flexible, with the ability to log certain events using
as little of 16 bytes of storage space. It also provides the ability to spend
more data and record more detail of program state at the time of the event, an
arbitrary string, or binary data.

\section{Terminology} \label{sec:terminology}

See table \ref{tab:terminology} for definitions of terms used throughout this
document. \textbf{TODO: There probably should be more here.}

\begin{table}[ht!]
    \begin{center}
        \begin{tabularx}{\textwidth}{|l|X|}
            \hline
            \emph{Term}           & \emph{Definition}                                                                          \\ \hline
            \textbf{Entry}        & A piece of data in the log recording the occurrence of a particular event                  \\ \hline
            \textbf{Log}          & A section of memory set aside to store data according to this standard                \\ \hline
            \textbf{Logger}       & A software utility which stores information to a log in accordance with this standard \\ \hline
            \textbf{Log Header}   & The section of the log allocated to storing log metadata                                   \\ \hline
        \end{tabularx}
        \caption{Definition of terms used within this document.}
        \label{tab:terminology}
    \end{center}
\end{table}

All stored numbers shall be in little-endian format. All stored addresses shall
be offsets from the beginning of the log memory. All lengths shall be in bytes.

\section{Log Header} \label{sec:lh}

\subsection{Description} \label{sec:lh:desc}

Each log shall begin with a data section encoding metadata relevant to the log
as a whole, referred to as the 'log header'. This log header shall reside at the
lowest addresses of the memory set aside for the log, and shall occupy 64 bytes.
If the memory used for logging only supports page-sized operation, the log
entries shall start at the next page boundary after the end of the header.

The log header shall contain the following information:

\begin{itemize}
    \item{Full git revision hash at the time of compilation}
    \item{Build date}
    \item{Flags:}
        \begin{itemize}
            \item{Dirty (D) -- shall be set if the git repository was 'dirty' at
                  build time}
              \item{Not Stable (\textoverline{S}) -- shall be set if the
                    firmware was not pinned at a stable release at build time.
                    Will never be cleared when D is set}
        \end{itemize}
    \item{Project ID}
    \item{Stable version (if D and \textoverline{S} flags are both cleared)}
    \item{Log data address}
\end{itemize}

The logger shall verify this information at each system reset and, if any has
changed, shall update the log header \emph{and remove all currently stored log
entries}.  \textbf{Everyone, is this desireable behavior?}

\subsection{Data Format} \label{sec:lh:df}

\newcommand{\magicnumberhx}{0x6C6F67}
\newcommand{\versionhx}{0x01}

See table \ref{tab:lh:df} for a description of data layout in memory.

The \emph{Project ID} field shall correspond to BES's internal project numbering
system. The major and minor firmware version numbers shall only be considered
valid if both the \emph{D} and \emph{\textoverline{S}} flags are cleared.

The \emph{Git Hash} field shall be the binary encoding of the full 40-character
hash associated with a git commit. Since this hash is a hexadecimal number, it
can be stored in 20 bytes of memory.

The \emph{Build Timestamp} field shall be a unix epoch timestamp indicating when
the software was compiled. It shall be stored in 64-bit form to avoid the 2030
rollover issue.

The \emph{Log Start} field is the address at which valid log data starts. This
may be the address immediately following the header (0x0000 0040), or it may be
the beginning of the next writeable page of memory.

The \emph{Log End} field is one more than the last valid address for the log
memory region. That is, if the last address in the log region is 0x0000 0FFF,
\emph{Log End} shall be 0x000 1000.

\begin{table}[ht!]
    \begin{center}
        \begin{bytefield}[endianness=big, bitwidth=\bfbitwidth, leftcurly=., rightcurlyspace=0pt]{32}
            \bitheader{0-31} \\
            \begin{leftwordgroup}{0x0000 0000}
                \bitbox{24}{Magic number (\magicnumberhx)} &
                \bitbox{8}{Spec Version (\versionhx)}
            \end{leftwordgroup} \\
            \begin{leftwordgroup}{0x0000 0004}
                \wordbox[tlr]{1}{}
            \end{leftwordgroup} \\
            \wordbox[lr]{3}{Git Hash} \\
            \begin{leftwordgroup}{0x0000 0014}
                \wordbox[blr]{1}{}
            \end{leftwordgroup} \\
            \begin{leftwordgroup}{0x0000 0018}
                \wordbox[tlr]{1}{\strut\smash{\lower 1em \hbox{Build Timestamp}}}
            \end{leftwordgroup} \\
            \begin{leftwordgroup}{0x0000 001C}
                \wordbox[blr]{1}{}
            \end{leftwordgroup} \\
           \begin{leftwordgroup}{0x0000 0020}
               \bitbox{1}{D} &
               \bitbox{1}{\textoverline{S}} &
               \bitbox{2}{\color{lightgray}\rule{\width}{\height}}
               \bitbox{12}{Project ID} &
               \bitbox{8}{Major version} &
               \bitbox{8}{Minor version}
           \end{leftwordgroup} \\
            \begin{leftwordgroup}{0x0000 0024}
                \wordbox{1}{Log Start}
            \end{leftwordgroup} \\
            \begin{leftwordgroup}{0x0000 0028}
                \wordbox{1}{Log End}
            \end{leftwordgroup} \\
            \begin{leftwordgroup}{0x0000 002C}
                \wordbox[tlr]{1}{}
            \end{leftwordgroup} \\
            \wordbox[lr]{3}{Reserved} \\
            \begin{leftwordgroup}{0x0000 003C}
                \wordbox[blr]{1}{}
            \end{leftwordgroup} \\
        \end{bytefield}
        \caption{Data format used to store the log header. Address values are
        offsets from log region base address.}
        \label{tab:lh:df}
    \end{center}
\end{table}

\section{Log Bookkeeping} \label{sec:lb}

\subsection{Description} \label{sec:lb:desc}

The log shall begin at the \emph{Log Start}, and shall extend to \emph{Log End}
(both stored in the log header, described in section \ref{sec:lh}).

The log shall function as a circular buffer. Log entries (section \ref{sec:le})
shall be inserted into available memory sequentially, aligned to word (4-byte)
boundaries.

\subsection{Data Format} \label{sec:lb:df}

See table \ref{tab:lb:df} for a description of how bookkeeping data are stored
in memory.

The first 16 bytes of the log shall be reserved for bookkeeping data. This
consists of two pointers, \emph{Head} and \emph{Tail}. \emph{Head} shall be the
address of the next valid location for insertion. \emph{Tail} shall be the
address of the oldest stored log entry, or shall be equal to \emph{Head} if no
log entries are stored. When the log is empty, both \emph{Head} and \emph{Tail}
shall be equal to \emph{Log Start} + 16. At no time will either head or tail be
less than \emph{Log Start} + 16 or greater than or equal to \emph{Log End}.

The bookkeeping data shall always reside at the address pointed to by \emph{Log
Start}.

New log entries shall be inserted at \emph{Head}.

When inserting a new log entry, the logger shall first verify that there is
space available. Space is available if the space between \emph{Head} and
\emph{Tail} (wrapped around the end of the memory region) is greater than
\emph{Entry Length}.

If there is no space available, the logger shall remove log entries by moving
\emph{Tail} to the start of the first entry where this condition is met.

The logger shall then write the log entry into memory at \emph{Insertion Point},
and shall set \emph{Head} to the next word beyond the end of the new log entry.

In order to avoid memory corruption, the logger shall always perform these steps
in the following order:

\begin{enumerate}
    \item{If needed, move \emph{Tail} to remove old log entries and make space for the new one}
    \item{Write the new log entry at \emph{Head}}
    \item{Move \emph{Head} to its new position}
\end{enumerate}

\begin{table}[ht!]
    \begin{center}
        \begin{bytefield}[endianness=big, bitwidth=\bfbitwidth, leftcurly=., rightcurlyspace=0pt]{32}
            \bitheader{0-31} \\
            \begin{leftwordgroup}{0x0000 0000}
                \wordbox{1}{Head}
            \end{leftwordgroup} \\
            \begin{leftwordgroup}{0x0000 0004}
                \wordbox{1}{Tail}
            \end{leftwordgroup} \\
            \begin{leftwordgroup}{0x0000 0008}
                \wordbox[tlr]{1}{\strut\smash{\lower 1em \hbox{Reserved}}}
            \end{leftwordgroup} \\
            \begin{leftwordgroup}{0x0000 000C}
                \wordbox[blr]{1}{}
            \end{leftwordgroup} \\
        \end{bytefield}
        \caption{Format used to record log bookkeeping data. Address values are
                 offsets from \emph{Log Start}.}
        \label{tab:lb:df}
    \end{center}
\end{table}

\section{Log Entry} \label{sec:le}

\subsection{Description} \label{sec:le:desc}

A log entry is a data structure which encodes the occurrence of a single event.
Unlike the other data formats, several fields are optional. At minimum, a log
shall include a length, a timestamp, and a type, which will together occupy 16
bytes.

\subsection{Data Format} \label{sec:le:df}

See table \ref{tab:le:df} for a description of a log entry's layout in memory.

The \emph{Entry Length} shall be the total length of the log entry (including
the \emph{Entry Length} field) in bytes. \emph{Entry Length} shall never be the
reserved value 0xFFFF FFFF.

The \emph{Timestamp} shall be either a UTC Unix epoch timestamp (if the \emph{R}
bit is cleared) or a relative (system time) timestamp from system reset (if the
\emph{R} bit is set). Absolute timestamps are preferred for systems which
include an RTC.

The \emph{Entry Type} is a field with standard log types defined in table
\ref{tab:le:et}. There is also a range of values set aside for arbitrary use.
Note that all (non-extended) ASCII characters fall into the user-defined range.

\begin{table}[ht!]
    \begin{center}
        \begin{tabularx}{\textwidth}{|l|l|X|}
            \hline
            \emph{Entry Type} & \emph{Field Value} & \emph{Notes}                                   \\ \hline
            User defined      & 0x00 - 0x7F        & Can be used to indicate user-defined log types \\ \hline
            Reset             & 0x80               &                                                \\ \hline
            Exception         & 0x81               &                                                \\ \hline
            Runtime Error     & 0x82               & Subtype will indicate the error type           \\ \hline
            State Change      & 0x83               & Subtype will indicate the \emph{entered} state \\ \hline
            Reserved          & 0x84 - 0xFF        & May be used in later versions of this standard \\ \hline
        \end{tabularx}
        \caption{Entry type codes}
        \label{tab:le:et}
    \end{center}
\end{table}

The \emph{Entry Subtype} field has meaning dependent on the particular entry type;
see table \ref{tab:le:est} for more details.

\begin{table}[ht!]
    \begin{center}
        \begin{tabularx}{\textwidth}{|l|l|l|X|}
            \hline
            \emph{Entry Type}               & \emph{Entry Subtype}      & \emph{Field Value} & \emph{Notes}                                                       \\ \hline
            \multirow{7}{*}{Reset}          & User defined              & 0x00-0x7F          &                                                                    \\ \cline{2-4}
            \textbf{}                       & Unknown Reset             & 0x80               & Can be used where subtype discrimination is not important          \\ \cline{2-4}
            \textbf{}                       & Power Reset               & 0x81               &                                                                    \\ \cline{2-4}
            \textbf{}                       & Software Reset            & 0x82               &                                                                    \\ \cline{2-4}
            \textbf{}                       & Watchdog Reset            & 0x83               &                                                                    \\ \cline{2-4}
            \textbf{}                       & Reserved                  & 0x84-0xFF          & May be used in later versions of this standard                     \\ \hline
            \multirow{8}{*}{Exception}      & User defined              & 0x00-0x7F          &                                                                    \\ \cline{2-4}
            \textbf{}                       & Unknown Exception         & 0x80               & Can be used where subtype discrimination is not important          \\ \cline{2-4}
            \textbf{}                       & Stack Overflow            & 0x81               &                                                                    \\ \cline{2-4}
            \textbf{}                       & Hard Fault                & 0x82               &                                                                    \\ \cline{2-4}
            \textbf{}                       & Bus Fault                 & 0x83               &                                                                    \\ \cline{2-4}
            \textbf{}                       & Usage Fault               & 0x84               &                                                                    \\ \cline{2-4}
            \textbf{}                       & Reserved                  & 0x85-0xFF          & May be used in later versions of this standard                     \\ \hline
            \multirow{10}{*}{Runtime Error} & User defined              & 0x00-0x7F          &                                                                    \\ \cline{2-4}
            \textbf{}                       & Unknown Runtime error     & 0x80               & Can be used where subtype discrimination is not important          \\ \cline{2-4}
            \textbf{}                       & Invalid Log Type          & 0x81               & Can be used by logger implementation to signal an invalid log call \\ \cline{2-4}
            \textbf{}                       & Invalid Log Subtype       & 0x82               & Can be used by logger implementation to signal an invalid log call \\ \cline{2-4}
            \textbf{}                       & Invalid Argument          & 0x83               &                                                                    \\ \cline{2-4}
            \textbf{}                       & Buffer Overflow           & 0x84               &                                                                    \\ \cline{2-4}
            \textbf{}                       & Memory Allocation Failure & 0x85               &                                                                    \\ \cline{2-4}
            \textbf{}                       & Reserved                  & 0x86-0xFF          & May be used in later versions of this standard                     \\ \hline
            \multirow{2}{*}{All others}     & User defined              & 0x00-0x7F          &                                                                    \\ \cline{2-4}
            \textbf{}                       & Reserved                  & 0x80-0xFF          & May be used in later versions of this standard                     \\ \hline
        \end{tabularx}
        \caption{Entry subtype codes}
        \label{tab:le:est}
    \end{center}
\end{table}

The bits \emph{PC}, \emph{SP}, \emph{SU}, \emph{DSC}, and \emph{BIN}
(collectively known as the '\emph{Field Bits}') indicate which optional fields
are present after the mandatory fields. Because these fields are optional, and
because some of the fields are variable length, their offsets are non-constant.
The logger shall place fields which are present  in the order they are listed in
table \ref{tab:le:df}, leaving no padding where optional fields are not present.
If no fields are present, the \emph{Field Bits} shall all be cleared and
\emph{Entry Length} shall be 16.

The \emph{BIN} member of the \emph{Field Bits} shall be set if one or more
\emph{Binary Data} fields are present in the log entry.

The \emph{Program Counter} field shall be used to record execution state at the
time of the entry. Note that this will be the value of the program counter
immediately before a call to any log function (usually retrieved by examining
the link register).

The \emph{Stack Pointer} field shall be used to record the immediate value of
the stack pointer at the time of the entry.

The \emph{Maximum Stack Usage} field shall be used to record the greatest amount
of stack allocated up to the time of the entry.

If this is used on a system with multiple threads (and hence multiple stack
pointers and program counters), it may be useful to omit the three preceding
fields and instead record the information in several \emph{Binary Data} fields.

The \emph{Descriptor String} variable-length field shall begin with a length
(\emph{Descriptor String Length}) which indicates the number of bytes in
\emph{Descriptor String Body} (including padding). The length shall always be
aligned to a word boundary (a multiple of 4), but the highest-order byte of the
length (\emph{DPB}) indicate the number of zero padding bytes at the end of the
data block.

The binary data section shall be made up of one or more \emph{Binary Data}
variable length fields (see description below). The \emph{Binary Data Count}
field indicates the number of subsequent \emph{Binary Data} entries.

Each \emph{Binary Data} variable-length field shall begin with a length
(\emph{Binary Data Length}) which indicates the number of bytes in \emph{Binary
Data Body} (including padding). The length shall always be aligned to a word
boundary (a multiple of 4), but the highest-order byte of the length
(\emph{BPB}) indicate the number of zero padding bytes at the end of the data
block. The \emph{Binary Data Type} field may be left zeroed, or may be used as a
user-defined discriminator between different data points.

\begin{table}[ht!]
    \begin{center}
        \begin{bytefield}[endianness=big, bitwidth=\bfbitwidth, leftcurly=., rightcurlyspace=0pt]{32}
            \bitheader{0-31} \\
            \begin{rightwordgroup}{Mandatory \\ Fields}
                \begin{leftwordgroup}{0x0000 0000}
                    \wordbox{1}{Entry Length}
                \end{leftwordgroup} \\
                \begin{leftwordgroup}{0x0000 0004}
                    \bitbox{1}{R} &
                    \bitbox[tlr]{31}{\strut\smash{\lower 1em \hbox{Timestamp}}}
                \end{leftwordgroup} \\
                \begin{leftwordgroup}{0x0000 0008}
                    \wordbox[blr]{1}{}
                \end{leftwordgroup} \\
                \begin{leftwordgroup}{0x0000 000C}
                    \bitbox{1}{\rotatebox{90}{\small PC}} &
                    \bitbox{1}{\rotatebox{90}{\small SP}} &
                    \bitbox{1}{\rotatebox{90}{\small SU}} &
                    \bitbox{1}{\rotatebox{90}{\small DSC}} &
                    \bitbox{1}{\rotatebox{90}{\small BIN}} &
                    \bitbox{11}{\color{lightgray}\rule{\width}{\height}} &
                    \bitbox{8}{Entry Type} &
                    \bitbox{8}{Entry Subtype}
                \end{leftwordgroup}
            \end{rightwordgroup} \\
            \begin{rightwordgroup}{Only if PC set}
                \wordbox{1}{Program Counter}
            \end{rightwordgroup} \\
            \begin{rightwordgroup}{Only if SP set}
                \wordbox{1}{Stack Pointer}
            \end{rightwordgroup} \\
            \begin{rightwordgroup}{Only if SU set}
                \wordbox{1}{Maximum Stack Usage}
            \end{rightwordgroup} \\
            \begin{rightwordgroup}{Descriptor String \\ Only if DSC set}
                \bitbox{8}{DPB} &
                \bitbox{8}{\color{lightgray}\rule{\width}{\height}} &
                \bitbox{16}{Descriptor String Length} \\
                \wordbox[tlr]{2}{Descriptor String Body} \\
                \bitbox[blr]{16}{} & \bitbox{16}{String Padding (0x00)}
            \end{rightwordgroup} \\
            \begin{rightwordgroup}{Binary Data \\ Only if BIN set}
                \wordbox{1}{Binary Data Count} \\
                \bitbox{8}{BPB} &
                \bitbox{8}{Binary Data Type} &
                \bitbox{16}{Binary Data Length} \\
                \wordbox[tlr]{2}{Binary Data Body} \\
                \bitbox[blr]{16}{} & \bitbox{16}{Data Padding (0x00)} \\
                \wordbox[]{1}{$\cdots$} \\
                \bitbox{8}{BPB} &
                \bitbox{8}{Binary Data Type} &
                \bitbox{16}{Binary Data Length} \\
                \wordbox[tlr]{2}{Binary Data Body} \\
                \bitbox[blr]{16}{} & \bitbox{16}{Data Padding (0x00)}
            \end{rightwordgroup}
        \end{bytefield}
        \caption{Format used to store a log entry.}
        \label{tab:le:df}
    \end{center}
\end{table}

\end{document}
